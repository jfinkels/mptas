\documentclass{article}

\usepackage{amsthm}
\usepackage{thmtools}
% Package `hyperref` must come before package `complexity`.
\usepackage[pdftitle={Clarification of a proof of completeness in classes of approximable optimization problems}, pdfauthor={Jeffrey Finkelstein}]{hyperref}
\usepackage{complexity}
\usepackage{algorithm}
\usepackage[noend]{algpseudocode}
\usepackage{amsmath}
\usepackage{amssymb}

\declaretheorem[numberwithin=section]{theorem}
\declaretheorem[numberlike=theorem]{lemma}
\declaretheorem[numberlike=theorem]{corollary}
\declaretheorem[numberlike=theorem, style=definition]{definition}
\declaretheorem[numberlike=theorem, style=definition]{todo}

%% http://tex.stackexchange.com/a/5255
%\DeclareMathOperator*{\argmax}{\arg\!\max}
\DeclareMathOperator*{\argmax}{arg\,max}
\newcommand{\ceil}[1]{\left\lceil#1\right\rceil}
\newcommand{\email}[1]{\href{mailto:#1}{\nolinkurl{#1}}}
\newcommand{\algorithmautorefname}{Algorithm}
\algrenewcommand\algorithmicrequire{\textbf{Input:}}

\author{Jef{}frey Finkelstein}
\date{\today}
\title{Clarification of a proof of completeness in classes of approximable optimization problems}

\begin{document}

\maketitle

\section{Introduction}
This work presents a more precise proof of a theorem in \cite{ep06} that provides sufficient conditions for an optimization problem to be hard for a class of approximable optimization problems under a certain type of reduction.
For consistency, we use definitions and notation that match and extend the definitions from that work.

\section{Preliminaries}
We denote the real numbers by $\mathbb{R}$ and the positive real numbers by $\mathbb{R}^+$.
Similarly we denote the rational numbers by $\mathbb{Q}$ and the positive rational numbers by $\mathbb{Q}^+$.
We denote the natural numbers (including 0) by $\mathbb{N}$.
For all $a$ and $b$ in $\mathbb{R}$ with $a < b$, we denote the open interval between $a$ and $b$ by $(a, b)$ and the closed interval by $[a, b]$.

If $S$ is a set, $S^*$ denotes the set of all finite sequences consisting of elements of $S$.
For each finite sequence $s$, we denote the length of $s$ by $|s|$.
We denote the set of all finite binary strings by $\Sigma^*$.

\begin{definition}\label{def:bigo}
  Suppose $f$ and $g$ are functions with domain $\mathbb{N}$ and codomain $\mathbb{R}$.
  We denote by $O(g(n))$ the set of all functions $f \colon \mathbb{N} \to \mathbb{R}$ such that there exists a $k \in \mathbb{Q}^+$ and an $N \in \mathbb{N}$ such that for all $n \in \mathbb{N}$ we have $n > N$ implies $f(n) \leq k g(n)$.
\end{definition}

We will use the following lemma to simplify the proof of the main theorem.

\begin{lemma}\label{lem:bigo}
  Suppose $f$ and $g$ are two functions with domain $\mathbb{N}$ and codomain $\mathbb{R}$.
  If $g(n) > 1$ for all sufficiently large $n$, then $f(n)$ is in $O(g(n)$ if and only if $f(n) - 1$ is in $O(g(n) - 1)$.
\end{lemma}

In this work we wish to determine the approximability of optimization problems.
Therefore, we require rigorous definitions for both optimization problems and approximations for them.

\begin{definition}
  An optimization problem is given by $(I, S, m, t)$, where $I$ is called the \emph{instance set}, $S \subseteq I \times \Sigma^*$ and is called the \emph{solution relation}, $m \colon I \times \Sigma^* \to \mathbb{N}$ and is called the \emph{measure function}, and $t\in\{\max, \min\}$ and is called the \emph{type} of the optimization.
  \begin{enumerate}
  \item The set of solutions corresponding to an instance $x \in I$ is denoted $S_P(x)$.
  \item The \emph{optimal measure} for any $x \in I$ is denoted $m^*(x)$.
    Observe that for all $x \in I$ and all $y \in \Sigma^*$, if $t = \max$ then $m(x, y) \leq m^*(x)$ and if $t = \min$ then $m(x, y) \geq m^*(x)$.
  \item The \emph{performance ratio}, $\rho$, of a solution $y$ for $x$ is defined by
    \begin{equation*}
      \rho(x, y) =
      \begin{cases}
        \frac{m(x, y)}{m^*(x)} & \text{if } t = \max \\
        \frac{m^*(x)}{m(x, y)} & \text{if } t = \min
      \end{cases}
    \end{equation*}
  \item A function $f \colon I \to \Sigma^*$ is an \emph{$r(n)$-approximator} if $\rho(x, f(x)) \geq \frac{1}{r(|x|)}$, for some function $r \colon \mathbb{N} \to \mathbb{Q}^+$.
  \end{enumerate}
\end{definition}

Just as a focus of computational complexity for decision problems is on the relationship between \P{} and \NP{}, so too do we study the relationship between optimization problems in \NP{}.

\begin{definition}
  The complexity class \NPO{} is the class of all optimization problems $(I, S, m, t)$ such that the following conditions hold.
  \begin{enumerate}
  \item The instance set $I$ is decidable by a deterministic polynomial time Turing machine.
  \item The solution relation $S$ is decidable by a deterministic polynomial time Turing machine and is polynomially bounded (that is, there is a polynomial $p$ such that for all $(x, y) \in S$, we have $|y| \leq p(|x|)$).
  \item The measure function $m$ is computable by a deterministic polynomial time Turing machine.
  \end{enumerate}
\end{definition}

Some optimization problems in \NPO{} are approximable within some ratio, so we wish to classify those problems as well.

\begin{definition}
  If $\mathcal{F}$ is a collection of functions, $\FAPX$ is the subclass of \NPO{} in which for each optimization problem $P$ there exists a polynomial time computable $r(n)$-approximator for $P$, where $r(n) \in \mathcal{F}$.
\end{definition}

In this notation, the class usually identified as \APX{} is written $O(1)\mhyphen\APX$.

The following definitions are required for producing complete problems for $\FAPX$.

\begin{definition}\label{def:downwardclosed}
  A collection of functions $\mathcal{F}$ is \emph{downward closed} if for all functions $g \in \mathcal{F}$, all $c \in \mathbb{R}^+$, and all functions $h$ we have $h(n) \in O(g(n^c))$ implies $h \in \mathcal{F}$.
\end{definition}

\begin{definition}\label{def:hardfunction}
  A function $g$ is \emph{hard} for the collection of functions $\mathcal{F}$ if for all $h \in \mathcal{F}$ there is a $c \in \mathbb{R}^+$ such that $h(n) \in O(g(n^c))$.
\end{definition}

\begin{definition}\label{def:gapintroducing}
  Suppose $L$ is a decision problem in \NP, $P$ is a maximization problem, and $\mathcal{F}$ is a downward closed collection of functions.
  There is an \emph{enhanced $\mathcal{F}$-gap-introducing reduction from $L$ to $P$} if there are functions $f \colon \left(\Sigma^* \times \mathbb{Q}^+ \right) \to I$ and $g \colon \left( \Sigma^* \times \Sigma^* \times \mathbb{Q}^+\right) \to \Sigma^+$, constants $n_0 \in \mathbb{R}^+$ and $c \in \mathbb{R}^+$, and a function $F \colon \mathbb{Q}^+ \to \mathbb{Q}^+$ such that
  \begin{enumerate}
  \item $F$ is hard for $\mathcal{F}$,
  \item $F$ is non-decreasing for all sufficiently large $n$,
  \item $F(n) > 1$ for all sufficiently large $n$,
  \end{enumerate}
  and for all $x \in \Sigma^+$ with $|x| \geq n_0$ and for all $N \in \mathbb{Q}^+$ such that $N \geq |x|^c$,
  \begin{enumerate}
  \item if $x \in L$ then $m_P^*(f(x, N)) \geq N$,
  \item if $x \notin L$ then $m_P^*(f(x, N)) \leq \frac{1}{F(N)} N$,
  \item for all $y \in S_P(f(x, N))$ such that $m_P(x, y) > \frac{1}{F(N)} N$, we have $g(x, y, N)$ is a witness that $x \in L$, and
  \item $f$, $g$, and $F$ are computable in deterministic time polynomial in $|x|$.
  \end{enumerate}
\end{definition}

The above definition can be modified to allow $P$ to be a minimization problem by swapping the direction of the inequalities and using $F(N) \cdot N$ instead of $\frac{1}{F(N)} N$.

\begin{definition}\label{def:canonicallyhard}
  Suppose $\mathcal{F}$ is a downward closed family of functions and suppose $P$ is a maximization problem in $\NPO$, with $P = (I, S, m, \max)$.
  The maximization problem $P$ is \emph{canonically hard} for $\FAPX$ if for all decision problems $L \in \NP$ there is an enhanced $\mathcal{F}$-gap-introducing reduction from $L$ to $P$.
\end{definition}

One of the main tools for studying the structural complexity of class of approximable optimization problems is the approximation preserving reduction.
One such reduction is the PTAS reduction.

\begin{definition}\label{def:ptasreduction}
  Suppose $P$ and $Q$ are maximization problems in $\NPO$ with $P = (I_P, S_P, m_P, \max)$ and $Q = (I_Q, S_Q, m_Q, \max)$.
  There is a \emph{PTAS reduction} from $P$ to $Q$ if there are functions $f \colon \left(I_P \times \mathbb{Q}^+\right) \to I_Q$, $g \colon \left(I_P \times \Sigma^* \right) \to \Sigma^*$, and $c \colon (0, 1) \to (0, 1)$ such that for all $x \in I_P$ and all $\epsilon \in (0, 1)$,
  \begin{enumerate}
  \item for all $y \in S_Q(f(x, \epsilon))$, we have $g(x, y, \epsilon) \in S_P(x)$,
  \item for all $y \in S_Q(f(x, \epsilon))$, if $\rho_Q(f(x, \epsilon), y) \geq 1 - c(\epsilon)$ then $\rho_P(x, g(x, y, \epsilon)) \geq 1 - \epsilon$, and
  \item $f$ and $g$ are computable in deterministic time polynomial in $|x|$.
  \end{enumerate}
\end{definition}

The PTAS reduction has been succesful in proving completeness in certain classes of approximable optimization problems.
In \cite{ep06}, the authors introduce the following relaxation of the PTAS reduction in order to allow more general completeness results.

\begin{definition}\label{def:mptasreduction}
  Suppose $P$ and $Q$ are maximization problems in $\NPO$ with $P = (I_P, S_P, m_P, \max)$ and $Q = (I_Q, S_Q, m_Q, \max)$.
  There is an \emph{MPTAS reduction} from $P$ to $Q$ if there are a polynomial $p$, functions $f \colon \left(I_P \times \mathbb{Q}^+\right) \to I_Q^*$, $g \colon \left(I_P \times {\left(\Sigma^*\right)}^* \right) \to \Sigma^*$, and $c \colon (0, 1) \to (0, 1)$ such that for all $x \in I_P$ and all $\epsilon \in (0, 1)$,
  \begin{enumerate}
  \item $|f(x, \epsilon)| \leq p(|x|)$,
  \item if $f(x, \epsilon) = (x_1', \dotsc, x_{p(|x|)}')$ then for all $\mathbf{y} \in (S_Q(x_1') \times \dotsb \times S_Q(x_{p(|x|)}'))$, we have $g(x, \mathbf{y}, \epsilon) \in S_P(x)$,
  \item if $f(x, \epsilon) = (x_1', \dotsc, x_{p(|x|)}')$ then for all $\mathbf{y} \in (S_Q(x_1') \times \dotsb \times S_Q(x_{p(|x|)}'))$ there is a $j \in \{1, \dotsc, p(|x|)\}$ such that $\rho_Q(x_j', y_j) \geq 1 - c(\epsilon)$ implies $\rho_P(x, g(x, \mathbf{y}, \epsilon)) \geq 1 - \epsilon$, where $\mathbf{y} = (y_1, \dotsc, y_{p(|x|)})$, and
  \item $f$ and $g$ are computable in deterministic polynomial time with respect to $x$.
  \end{enumerate}
\end{definition}

The ``M'' in ``MPTAS'' stands for ``multi-valued'', because $f$ produces a finite sequence of instances instead of a single instance, as in the PTAS reduction.

\begin{lemma}\label{lem:ptasimpliesmptas}
  Suppose $P$ and $Q$ are two optimization problems.
  If there is a PTAS reduction from $P$ to $Q$ then there is an MPTAS reduction from $P$ to $Q$.
\end{lemma}
\begin{proof}
  Suppose $(f, g, c)$ is the PTAS reduction from $P$ to $Q$.
  If we consider the output of $f$ and the second input to $g$ to be sequences of length one, then $(f, g, c)$ are also an MPTAS reduction from $P$ to $Q$.
\end{proof}

The proof of the next lemma is straightforward but tedious, so it is omitted.

\begin{lemma}\label{lem:composition}
  MPTAS reductions are closed under composition.
\end{lemma}

\section{Results}

\begin{theorem}[{\cite[Theorem~2]{ep06}}]\label{thm:hardness}
  Suppose $\mathcal{F}$ is a class of downward closed functions and $Q$ is a maximization problem.
  If $Q$ is canonically hard for $\FAPX$ then $Q$ is hard for the class of maximization problems in $\FAPX$ under MPTAS reductions.
\end{theorem}
\begin{proof}
  Suppose $P$ is a maximization problem in $\FAPX$, where $P$ is defined by $P = (I_P, S_P, m_P, \max)$.
  By the definition of $\FAPX$, we know the instance set $I_P$ is in \P, the solutions set $S_P$ is in \P{} and is polynomially bounded, the measure function $m_P$ is computable in deterministic polynomial time, and there is an $r(n)$-approximator $A$ for $P$ for some $r \in \mathcal{F}$.

  Assume without loss of generality that $m_P(x, y) \geq 1$ for all $x$ and $y$ (if $m_P(x, y) < 1$ for some $x$ and $y$, just consider a new maximization problem whose measure function is $m'_P(x, y) = m_P(x, y) + 1$; the approximation results will be the same).
  Since $m_P$ is computable in deterministic polynomial time, the \emph{length} of its output is bounded by a polynomial $p$ with respect to the length of its first input, $x$, so the \emph{value} of $m_P(x, y)$ is in the interval $[1, 2^{p(|x|)}]$.
  Our goal is to construct functions $f$, $g$, and $c$ satisfying the conditions of the MPTAS reduction.
  We do this by partitioning the interval $[1, 2^{p(|x|)}]$ into subintervals, each of ``multiplicative size'' $\frac{1}{1 - \epsilon}$, in order to find a solution that is within a $\frac{1}{1 - \epsilon}$ multiplicative factor of the optimal solution, then using the enhanced $\mathcal{F}$-gap-introducing reduction to reconstruct a candidate solution.

  For any $\epsilon \in (0, 1)$, consider the subintervals $\left[\left(\frac{1}{1 - \epsilon}\right)^{i - 1}, \left(\frac{1}{1 - \epsilon}\right)^i\right]$ for each $i \in \{1, \dotsc, M\}$ where $M = \ceil{\frac{p(|x|)}{\log_2{\frac{1}{1 - \epsilon}}}}$.
  We choose $M$ this way so that $\left(\frac{1}{1 - \epsilon}\right)^M \geq 2^{p(|x|)} \geq m_P^*(x)$ for all $x \in I_P$.
  Since $\epsilon$ is a constant independent of $x$, the value of $M$ is a polynomial in $|x|$.

  Define $L_i = \left\{ x \in I_P \,\middle|\, m_P^*(x) \geq \left(\frac{1}{1 - \epsilon}\right)^{i - 1}\right\}$ for each $i \in \{1, \dotsc, M\}$.
  Each of the languages $L_i$ is in \NPO: the witness is a candidate solution $y$ for $x$ and the verification procedure checks (in polynomial time) that $m_P(x, y) \geq \left(\frac{1}{1 - \epsilon}\right)^{i - 1}$ (using the fact that $m_P^*(x) \geq m_P(x, y)$ for all $y$).
  By hypothesis there is an enhanced $\mathcal{F}$-gap-introducing reduction from $L_i$ to $Q$ for each $i \in \{1, \dotsc, M\}$; let $f_i$, $g_i$, $F_i$, $n_i$, $c_i$ be the functions and constants satisfying the definition of that reduction.
  For each $i \in \{1, \dotsc, M\}$, since $r \in \mathcal{F}$ and $F_i$ is hard for $\mathcal{F}$, we have $r(n) \in O(F_i(n^{d_i}))$ for some $d_i \in \mathbb{R}^+$.
  In order to facilitate some inequalities in the proof below, we invoke \autoref{lem:bigo} to yield the equivalent statement $r(n) - 1 \in O(F_i(n^{d_i}) - 1)$.
  More specifically, there is a $k \in \mathbb{Q}^+$ and an $N \in \mathbb{N}$ such that for all $n \in \mathbb{N}$ with $n > N$, we have $r(n) - 1 \leq k(F_i(n^{d_i}) - 1)$, or in other words,
  \begin{equation}\label{eq:0}
    r(n) \leq k(F_i(n^{d_i}) - 1) + 1.
  \end{equation}
  
  At this point we can construct the MPTAS reduction from $P$ to $Q$.
  Let $f(x, \epsilon) = (f_1(x, N_1), \dotsc, f_M(x, N_M))$ for all $x \in I_P$ and all $\epsilon \in (0, 1)$, where $N_i = |x|^{\max(c_i, d_i)}$ for each $i \in \{1, \dotsc, M\}$.
  Define $g$ as in \autoref{alg:g}.
  \begin{algorithm}[t]
    \caption{Deterministic polynomial time algorithm that computes a $(1 - \epsilon)$-approximate solution for $x$ \label{alg:g}}
    \begin{algorithmic}
      \Require{$x \in I_P$, $(y_1, \dotsc, y_M) \in (S_Q(f_1(x, N_1)) \times \dotsb \times S_Q(f_M(x, N_M)))$, $\epsilon \in (0, 1)$}
      \Statex{}
      \Function{$g$}{$x$, $\mathbf{y}$, $\epsilon$}:
        \For{$i \in \{1, \dotsc, M\}$}
          \State{$N_i \gets |x|^{\max(c_i, d_i)}$}
          \If{$m_Q(f_i(x, N_i), y_i) > \frac{N_i}{F_i(N_i)}$}
            \State{$z_i \gets g_i(x, y_i, N_i)$}
          \Else
            \State{$z_i \gets A(x)$}
          \EndIf
        \EndFor
        \State{$Z \gets \{z_1, \dotsc, z_M\}$}
        \State{$z_j \gets \argmax_{z_i \in Z}\{m_P(x, z_i)\}$} \\
        \hspace{1.5em}\Return{$z_j$}
      \EndFunction
    \end{algorithmic}
  \end{algorithm}
  Let $c(\epsilon) = \frac{\epsilon}{\epsilon + k(\epsilon - 1)}$.

  Condition 1 of the reduction is satisfied because $M$ is bounded above by a polynomial in $|x|$, as previously stated.
  Condition 2 of the reduction is satisfied because $g$ outputs either $A(x)$ or $g_i(x, y_i, N_i)$ for some $i$.
  The output of the approximation algorithm $A$ when run on input $x$ is a solution in $S_P(x)$ by hypothesis.
  For sufficiently long $x$, the value of $g_i(x, y_i, N_i)$ is a witness that $x \in L_i$, or equivalently, a solution in $S_P(x)$.
  The value of $N_i$ is at least $|x|^c_i$, so $N_i$ is large enough that $g_i$ can reconstruct a witness.
  Condition 4 is satisfied because $f$ is computable in polynomial time as long as each $f_i$ is, and $g$ is computable in polynomial time as long as $A$, $m_Q$, $f_i$, $g_i$, and $F_i$ are (all other operations in $g$ can be performed in polynomial time).
  It remains to verify condition 3, the approximation preservation condition.

  Let $x \in I_P$, let $\epsilon \in (0, 1)$, and let $\mathbf{y} \in (S_Q(f_1(x, N_1)) \times \dotsb \times S_Q(f_M(x, N_M)))$, where $\mathbf{y} = (y_1, \dotsc, y_M)$.
  Assume without loss of generality that $x$ has a solution in $P$ (if $x$ had no solution in $P$, an approximation preserving reduction is meaningless anyway).
  Thus $m_P^*(x)$ exists and there must exist a $j \in \{1, \dotsc, M\}$ such that
  \begin{equation}\label{eq:1}
    \left(\frac{1}{1 - \epsilon}\right)^{j - 1} \leq m_P^*(x) \leq \left(\frac{1}{1 - \epsilon}\right)^j.
  \end{equation}
  We will show that this $j$ satisfies condition 3.
  Suppose $\rho_Q(f_j(x, N_j), y_j) \geq 1 - c(\epsilon)$.
  If $m_Q(f_j(x, N_j), y_j) > \frac{N_j}{F_j(N_j)}$ then $g(x, \mathbf{y}, \epsilon)$ outputs $g_j(x, y_j, N_j)$.
  By the construction of $g$ and the definition of $g_j$,
  \begin{equation}\label{eq:2}
    m_P(x, g_j(x, y_j, N_j)) \geq \left(\frac{1}{1 - \epsilon}\right)^{j - 1}.
  \end{equation}
  Combining \autoref{eq:1} and \autoref{eq:2} we have
  \begin{equation*}
    \left(\frac{1}{1 - \epsilon}\right)^{j - 1} \leq m_P(x, g_j(x, y_j, N_j)) = m_Q(x, g(x, \mathbf{y}, \epsilon)) \leq m_P^*(x) \leq \left(\frac{1}{1 - \epsilon}\right)^j,
  \end{equation*}
  which implies
  \begin{equation*}
    \rho_P(x, g(x, \mathbf{y}, \epsilon)) = \frac{m_P(x, g(x, \mathbf{y}, \epsilon))}{m_P^*(x)} \geq 1 - \epsilon.
  \end{equation*}
  (This is true even without the premise $\rho_Q(f_j(x, N_j), y_j) \geq 1 - c(\epsilon)$.)

  Suppose now that $m_Q(f_j(x, N_j), y_j) \leq \frac{N_j}{F_j(N_j)}$, so $g(x, \mathbf{y}, \epsilon)$ will yield $A(x)$.
  Since
  \begin{equation*}
    \rho_P(x, g(x, \mathbf{y}, \epsilon)) = \frac{m_P(x, g(x, \mathbf{y}, \epsilon))}{m_P^*(x)} = \frac{m_P(x, A(x))}{m_P^*(x)} \geq \frac{1}{r(|x|)},
  \end{equation*}
  it suffices to show $\frac{1}{r(|x|)} \geq 1 - \epsilon$.
  Since $m_P^*(x) \geq (\frac{1}{1 - \epsilon})^{j - 1}$, we have $x \in L_j$.
  Since $f_j$ is part of a gap-introducing reduction from $L_j$ to $Q$, we have $m_Q^*(f_j(x, N_j)) \geq N_j$.
  Hence
  \begin{equation*}
    \rho_Q(f_j(x, N_j), y_j) = \frac{m_Q(f_j(x, N_j), y_j)}{m_Q^*(f_j(x, N_j))} \leq \frac{\frac{N_j}{F_j(N_j)}}{N_j} = \frac{1}{F_j(N_j)}.
  \end{equation*}
  This implies $\frac{1}{F_j(N_j)} \geq 1 - c(\epsilon)$, or $F_j(N_j) \leq \frac{1}{1 - c(\epsilon)}$.
  By construction, $N_j = |x|^{\max(c_j, d_j)}$, so $N_j \geq |x|^{d_j}$.
  Since $F_j$ is non-decreasing for sufficiently large inputs, $F_j(N_j) \geq F_j(|x|^{d_j})$.
  Thus $F_j(|x|^{d_j}) \leq \frac{1}{1 - c(\epsilon)}$.
  Combining this with \autoref{eq:0} yields $r(|x|) \leq k\left(\frac{1}{1 - c(\epsilon)} - 1\right) + 1$, for sufficiently long $x$.
  Substituting the definition of $c(\epsilon)$ and performing some algebraic manipulation yields $r(|x|) \leq \frac{1}{1 - \epsilon}$, which is equivalent to $\frac{1}{r(|x|)} \geq 1 - \epsilon$.
  Therefore $\rho_P(x, g(x, \mathbf{y}, \epsilon)) \geq 1 - \epsilon$, completing the proof that $(f, g, c)$ is a correct MPTAS reduction from an arbitrary language $P$ in $\FAPX$ to $Q$.
\end{proof}

In some steps of the above proof we ignore strings $x$ that were not ``sufficiently long''.
This is acceptable because a lookup table of finite size can store the optimal solutions to all instances of optimization problem $P$ of length up to $n_0$, where $n_0$ is a constant (representing the ``sufficient length'').
The function $g$ could then look up and output the optimal solution for all inputs $x$ of length up to $n_0$.

\begin{corollary}[{\cite[Theorem~3]{ep06}}]
  Suppose $\mathcal{F}$ is a class of downward closed functions and suppose $Q$ is an optimization problem.
  If $Q$ is canonically hard for $\FAPX$ then $Q$ is hard for $\FAPX$ under MPTAS reductions.
\end{corollary}
\begin{proof}
  We already know from \autoref{thm:hardness} that there is an MPTAS reduction from every maximization problem in $\FAPX$ to $Q$.
  Consider an arbitrary minimization problem $P$ in $\FAPX$.
  By \cite{ep06}, there is an ``E reduction'' (another type of approximation preserving reduction) from $P$ to $P'$ for some maximization problem $P'$, and an E reduction implies a PTAS reduction (see, for example, \cite[Figure~2]{crescenzi97}).
  By \autoref{lem:ptasimpliesmptas}, a PTAS reduction implies an MPTAS reduction.
  By \autoref{lem:composition}, an MPTAS reduction from $P$ to $P'$ and another from $P'$ to $Q$ implies an MPTAS reduction from $P$ to $Q$.
  Therefore $Q$ is hard for $\FAPX$ under MPTAS reductions.
\end{proof}

\section*{About this work}

Copyright 2013 Jef{}frey Finkelstein.

This work is licensed under the Creative Commons Attribution-ShareAlike License 3.0.
Visit \mbox{\url{https://creativecommons.org/licenses/by-sa/3.0/}} to view a copy of this license.

The \LaTeX{} markup which generated this document is available on the World Wide Web at \mbox{\url{https://github.com/jfinkels/mptas}}.
It is also licensed under the Creative Commons Attribution-ShareAlike License.

The author can be contacted via email at \email{jeffreyf@bu.edu}.

\bibliographystyle{plain}
\bibliography{references}

\end{document}
